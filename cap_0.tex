% !TEX root=index.tex
\clearpage{\pagestyle{empty}\cleardoublepage}
\chapter{Introduzione}\label{ch:introduzione}
\markboth{Introduzione}{Introduzione}

Al giorno d’oggi è impossibile non aver sentito parlare almeno una volta di smartphone. Letteralmente \emph{cellulari intelligenti},  
riescono ad eseguire una serie di operazioni, anche molto complesse, che un telefono cellulare, nella sua accezione storica (ovvero dispositivo in grado di connettersi alla rete cellulare radiomobile), non sarebbe mai stato in grado di gestire. Stiamo parlando di funzionalità che vanno dalla più semplice consultazione di posta elettronica o navigazione di una pagina web sino ad arrivare alla registrazione di video in slow motion o permettere la navigazione satellitare.

Tutte queste operazioni sono possibili innanzitutto grazie a capacità di calcolo e memorie paragonabili a quelle montate nei moderni sistemi di calcolo desktop. Si pensi ad esempio agli ultimi modelli di smartphone dotati di processori octa-core e con banchi di RAM che sfiorano i 3GB.

Ad affiancare e controllare la grande potenza dell’hardware ci deve essere necessariamente una "grande mente": il sistema operativo, quello che potremmo definire il direttore d’orchestra del dispositivo, grazie al quale possiamo decidere se scattare una fotografia o rifiutare o meno una chiamata. \\*
Insomma, il sistema operativo crea un’\emph{interfaccia}, una sorta di ambiente virtuale, che permette all’utente di interagire con la macchina. 

Detto questo, è chiaro come un sistema operativo di uno smartphone, in particolar modo quella componente che definisce l’interfaccia grafica (in gergo GUI), non possa avere le stesse caratteristiche di quello di un desktop computer.
Ricordiamo infatti che stiamo parlando di dispositivi che nella maggior parte dei casi hanno dimensioni non superiori a quelle di un portafogli, che non restano fermi su una scrivania per il 99\% del loro ciclo di vita: vengono trasportati a destra e a manca sfruttando il loro grande punto di forza, la portabilità. 
Non è pensabile che su uno smartphone si possa chiudere un’applicazione cliccando su un bottone grande quanto una lenticchia o che si possa interagire con gli altri componenti utilizzando un puntatore come quello di un mouse.

Nonostante queste considerazioni possano sembrare ormai banali e scontate ai più, il merito di quello che noi oggi consideriamo la normalità va attribuito in primis ad un’azienda: la Apple. 
Sebbene l’idea di creare un dispositivo che unisse la telefonia con le capacità di un computer fosse risalente all’ormai lontano 1973, la storia degli smartphone per il mercato di massa comincia nel molto più vicino 2007, con il successo del primo iPhone. \\*
In quell’anno l’azienda di Cupertino sfidò il mercato mondiale con un dispositivo realizzato seguendo la filosofia \emph{user-friendly}. Un dispositivo che faceva della sua punta di diamante non tanto le caratteristiche tecniche quanto la semplicità di utilizzo. Fu un successo e nel giro di pochi anni l’iPhone divenne il prodotto di riferimento del mercato degli smartphone. La Apple era riuscita a dimostrare come l’esperienza utente fosse, ed è tuttora, uno dei principali fattori a determinare la conquista del settore.

Da allora lo sviluppo di applicazioni per dispositivi mobile è uno dei settori maggiormente in crescita nel campo dell’IT (Information Technology), tanto che nel 2009 la stessa Apple annunciò il raggiungimento di 2 miliardi di download di applicazioni, decretando il grande successo di iOS, il sistema operativo dei suoi smartphone. \\*
Ad oggi l’App Store di iPhone e il Play Store (la controparte di Android) si contendono il primato con un numero di applicazioni disponibili al download che si avvicina al milione e mezzo.

In questo contesto si colloca il lavoro di tesi, proponendo lo studio delle principali dinamiche che sono entrate in gioco nella progettazione e nello sviluppo dell’applicazione per iOS di Unifacile.

\section{Panoramica del servizio}\label{sec:panoramica_servizio}
	\begin{figure}[H]
		\centering
		\includegraphics[keepaspectratio=true,scale=0.4]{img/logo-univpm.png}
		\caption{Il logo di Unifacile\copyright}
		\label{fig:logo_univpm}
	\end{figure}

	L’applicazione sviluppata durante i mesi di tirocinio consiste in un adattamento per smartphone del portale Unifacile (figura \ref{fig:logo_univpm}), attualmente presente soltanto nella sua versione desktop.

	Unifacile è un \emph{social network} per universitari che viene ideato e sviluppato interamente da un gruppo di studenti di Ingegneria Informatica e dell’Automazione dell’Università Politecnica delle Marche, compreso il sottoscritto.

	Il sito, disponibile al pubblico a partire da settembre 2014, nasce con l’intento di soddisfare le tipiche esigenze di un universitario creando una piattaforma modellata interamente attorno lo studente. Domande fatte in sede d’esame, appunti, esercizi e pareri sugli insegnamenti sono solo una parte delle informazioni che vengono condivise su Unifacile dagli studenti. Al suo interno, infatti, è anche possibile trovare un elenco dei docenti e dei rispettivi corsi, compilare il proprio libretto universitario, consultare l’orario delle lezioni e molto altro ancora.

	Tutte le informazioni sono accessibili soltanto previa registrazione gratuita, durante la quale vengono richiesti, oltre ai classici nome, cognome ed email, anche informazioni legate alla vita accademica come università, corso di laurea, anno di immatricolazione. Queste ultime, in particolare, sono obbligatorie soltanto per permettere al sistema di generare automaticamente una serie di informazioni. Ad esempio, il piano di studi dello studente viene popolato una volta effettuato l’accesso, senza ricorrere alla compilazione manuale.

	La semplicità e la qualità delle informazioni presenti sono proprio gli aspetti cruciali che contraddistinguono Unifacile da tutti gli altri portali che cercano di fornire un servizio simile. 
	La grande maggioranza dei servizi, se non addirittura la totalità, preferisce infatti puntare sull’acquisizione di un grande parco utenti mediante l’utilizzo di stratagemmi che intaccano fortemente la validità delle informazioni e, necessariamente, anche quella del servizio stesso.
	Un esempio tipico consiste nel popolare la banca dati basandosi esclusivamente sulle informazioni fornite dagli utenti durante l’interazione con il sito; meccanismo che porta non solo ad una perdita di consistenza dei dati (lo stesso docente potrebbe essere inserito due volte a causa di errori ortografici), ma anche ad un'inaffidabilità dell’informazione non trascurabile (un corso potrebbe essere introvabile a causa di un errore di battitura da parte di uno studente).

	Proprio per evitare questo tipo di inconvenienti, Unifacile ha preferito raccogliere e controllare personalmente le informazioni necessarie al popolamento della banca dati, a discapito della rapidità di estensione del servizio ai vari atenei.
	Attualmente infatti il servizio è disponibile soltanto per una cerchia ristretta di utenti (le facoltà di Ingegneria, Economia e Medicina dell'Università Politecnica delle Marche), anche se ha come obiettivo primario quello di aprirsi a tutte le università italiane.
